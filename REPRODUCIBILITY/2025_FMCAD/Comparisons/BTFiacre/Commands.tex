\newcommand{\SwapArrow}{\rightleftharpoons{}}
\newcommand{\NodeStatus}{\texttt{\itshape{N}}}


\newcommand{\CurrentChapter}{Null}
\newcommand{\CurrentSection}{Null}
\newcommand{\CurrentSubsection}{Null}
\newcommand{\CurrentSubsubsection}{Null}
\newcommand{\CurrentParagraph}{Null}



\newcommand{\LabelChapter}[1]{#1}
\newcommand{\LabelSection}[2]{#1::#2}
\newcommand{\LabelSubsection}[3]{#1::#2::#3}
\newcommand{\LabelSubsubsection}[4]{#1::#2::#3::#4}
\newcommand{\LabelParagraph}[5]{#1::#2::#3::#4::#5}
\newcommand{\LabelObject}[6]{#1::#2::#3::#4::#5::#6}
\newcommand{\LabelChapterHere}[1]{\LabelChapter{#1}}
\newcommand{\LabelSectiontHere}[1]{\LabelSection{\CurrentChapter}{#1}}
\newcommand{\LabelSubsectionHere}[1]{\LabelSubsection{\CurrentChapter}{\CurrentSection}{#1}}
\newcommand{\LabelSubsubsectionHere}[1]{\LabelSubsubsection{\CurrentChapter}{\CurrentSection}{\CurrentSubsection}{#1}}
\newcommand{\LabelParagraphHere}[1]{\LabelParagraph{\CurrentChapter}{\CurrentSection}{\CurrentSubsection}{\CurrentSubsubsection}{#1}}
\newcommand{\LabelObjectHere}[1]{\LabelObject{\CurrentChapter}{\CurrentSection}{\CurrentSubsection}{\CurrentSubsubsection}{\CurrentParagraph}{#1}}
\newcommand{\NewSection}[1]{\renewcommand{\CurrentSection}{#1}\renewcommand{\CurrentSubsection}{Null}\renewcommand{\CurrentSubsubsection}{Null}\renewcommand{\CurrentParagraph}{Null}\section{#1}\label{\LabelSection{\CurrentChapter}{#1}}\pdfbookmark[1]{#1}{\LabelSection{\CurrentChapter}{#1}}}
\newcommand{\NewSubsection}[1]{\renewcommand{\CurrentSubsection}{#1}\renewcommand{\CurrentSubsubsection}{Null}\renewcommand{\CurrentParagraph}{Null}\subsection{#1}\label{\LabelSubsection{\CurrentChapter}{\CurrentSection}{#1}}\pdfbookmark[2]{#1}{\LabelSubsection{\CurrentChapter}{\CurrentSection}{#1}}}
\newcommand{\NewSubsubsection}[1]{\renewcommand{\CurrentSubsubsection}{#1}\renewcommand{\CurrentParagraph}{Null}\subsubsection{#1}\label{\LabelSubsubsection{\CurrentChapter}{\CurrentSection}{\CurrentSubsection}{#1}}\pdfbookmark[3]{#1}{\LabelSubsubsection{\CurrentChapter}{\CurrentSection}{\CurrentSubsection}{#1}}}
\newcommand{\NewParagraph}[1]{\renewcommand{\CurrentParagraph}{#1}\paragraph{#1}\label{\LabelParagraph{\CurrentChapter}{\CurrentSection}{\CurrentSubsection}{\CurrentSubsubsection}{#1}}\pdfbookmark[4]{#1}{\LabelParagraph{\CurrentChapter}{\CurrentSection}{\CurrentSubsection}{\CurrentSubsubsection}{#1}}}
% probably require booktabs and bookmark
%\newtheorem{sereneDefinition}{Definition}[section]
\newcommand{\CreateDefinition}[2]{\begin{definition}\label{\LabelObjectHere{#1}}\textbf{#1.} #2\end{definition}}
%requires package: color
\renewcommand\UrlFont{\color{blue}\rmfamily}

\newcommand{\AbsoluteValue}[1]{\lvert#1\rvert}
\newcommand{\Cardinality}[1]{\lvert#1\rvert}
\newcommand{\BuildFrom}{\mid}
\newcommand{\Integers}{\mathbb{Z}}
\newcommand{\SuchThat}{\text{ s.t. }}
%\newcommand{\Tuple}[1]{\mathcal{<}#1\mathcal{>}}
\newcommand{\Tuple}[1]{(#1)}
\newcommand{\DefinedAs}{\triangleq}
\newcommand{\True}{\top}
\newcommand{\False}{\bot}

\newcommand{\Union}{\cup}
\newcommand{\Multiply}{\cdot}

\newcommand{\Equivalent}{\equiv}

\newcommand{\EmptySet}{\varnothing}
\newcommand{\Biconditional}{\iff}


\newcommand{\Increment}{\mathbin{{+}{+}}}
\newcommand{\Decrement}{\mathbin{{-}{-}}}
\newcommand{\PlusEq}{\mathrel{+}=}

\newcommand{\Floor}[1]{\lfloor #1 \rfloor}

\newcommand{\Wildcard}{\_}

\newcommand{\Todo}[1]{\textcolor{red}{TODO: #1}}
\newcommand{\TODO}[1]{\Todo{#1}}
%start of Behavior-Tree.tex
\newcommand{\BehaviorTree}{\mathit{BT}}
\newcommand{\Invalid}{\texttt{\itshape{I}}}
\newcommand{\Failure}{\texttt{\itshape{F}}}
\newcommand{\Running}{\texttt{\itshape{R}}}
\newcommand{\Success}{\texttt{\itshape{S}}}

\definecolor{BehaviorTreeBlackboardColor}{HTML}{0000FF} %blue
\definecolor{BehaviorTreeEnvironmentColor}{HTML}{FF8C00} %dark orange
\newcommand{\EnvVar}[1]{\texttt{\itshape{\textcolor{BehaviorTreeEnvironmentColor}{#1}}}}
\newcommand{\BlVar}[1]{\texttt{\itshape{\textcolor{BehaviorTreeBlackboardColor}{#1}}}}

% requires tikz, tikz-qtree

\usetikzlibrary{shapes.geometric, shapes.misc, shapes.arrows, shapes.callouts, shapes.multipart, fit, positioning, calc, matrix, automata}

\definecolor{BehaviorTreeSelectorColor}{HTML}{00FFFF}
\definecolor{BehaviorTreeSequenceColor}{HTML}{FFA500}
\definecolor{BehaviorTreeParallelColor}{HTML}{FFD700}
\definecolor{BehaviorTreeLeafColor}{HTML}{C0C0C0}

\tikzset{
  Selector/.style={
    draw=black,
    fill=BehaviorTreeSelectorColor,
    text=black,
    shape=chamfered rectangle,
    minimum size=15pt,
    inner sep=0pt,
    font=\tiny
  },
  Sequence/.style={
    draw=black,
    fill=BehaviorTreeSequenceColor,
    text=black,
    shape=rectangle,
    minimum size=15pt,
    inner sep=0pt,
    font=\tiny
  },
  Parallel/.style={
    draw=black,
    fill=BehaviorTreeParallelColor,
    text=black,
    shape=trapezium,
    trapezium left angle=60,
    trapezium right angle=120,
    minimum size=15pt,
    inner sep=0pt,
    trapezium stretches body,
    font=\tiny,
    align=center
  },
  Decorator/.style={
    draw=black,
    fill=white,
    text=black,
    shape=trapezium,
    trapezium left angle=67,
    trapezium right angle=67,
    minimum size=15pt,
    inner sep=0pt,
    trapezium stretches body,
    font=\tiny,
    align=center
  },
  Action/.style={
    draw=black,
    fill=BehaviorTreeLeafColor,
    text=black,
    shape=ellipse,
    minimum size=15pt,
    inner sep=0pt,
    font=\tiny
  },
  Check/.style={
    draw=black,
    fill=BehaviorTreeLeafColor,
    text=red,
    shape=ellipse,
    minimum size=15pt,
    inner sep=0pt,
    font=\tiny
  },
  Blackboard/.style={
    draw=blue,
    fill=white,
    text=black,
    shape=rectangle,
    font=\tiny,
    inner sep = .5pt,
    inner xsep = -3pt,
    anchor = north
  }
}
  % ,
  % TreeTable/.style={
  %   matrix of nodes,
  %   row sep=0pt,
  %   column sep=0pt,
  %   % row sep=-\pgflinewidth,
  %   % column sep=-\pgflinewidth,
  %   nodes={
  %     rectangle,
  %     draw=black,
  %     align=center
  %   },
  %   % minimum height=1.5em,
  %   minimum height=1pt,
  %   text depth=0.5ex,
  %   text height=1.5ex,
  %   nodes in empty cells,
  %   column 1/.style={
  %     nodes={text width=5.0em,font=\bfseries}
  %   }
  % },
  % TreeTableWide/.style={
  %   matrix of nodes,
  %   row sep=0pt,
  %   column sep=0pt,
  %   % row sep=-\pgflinewidth,
  %   % column sep=-\pgflinewidth,
  %   nodes={
  %     rectangle,
  %     draw=black,
  %     align=center
  %   },
  %   % minimum height=1.5em,
  %   minimum width=7pt,
  %   minimum height=1pt,
  %   text depth=0.5ex,
  %   text height=1.5ex,
  %   nodes in empty cells,
  %   column 1/.style={
  %     nodes={text width=5.0em,font=\bfseries}
  %   }
  % }
% requires package cite
\newcommand{\CiteAsText}[1]{\cite{#1}}
\newcommand{\CiteAsRef}[1]{\cite{#1}}
\newcommand{\BiggestFishVar}{\BlVar{f}}

\newcommand{\NoOpt}{\textbf{no\_opt}}
\newcommand{\LastOpt}{\textbf{last\_opt}}
\newcommand{\FirstOpt}{\textbf{first\_opt}}
\newcommand{\FullOpt}{\textbf{full\_opt}}
\newcommand{\FiniteStateMachine}{\mathit{FSM}}
\newcommand{\FSMState}{S_{FSM}}
\newcommand{\FSMInitial}{s_{FSM}}
\newcommand{\FSMAlphabet}{\Sigma_{FSM}}
\newcommand{\FSMTransition}{\delta_{FSM}}

\newcommand{\LTLFormula}{\phi}
\newcommand{\LTLFormulaA}{\phi_1}
\newcommand{\LTLFormulaB}{\phi_2}
\newcommand{\LTLAtom}{a}
\newcommand{\LTLNext}{\bigcirc}
\newcommand{\LTLUntil}{\mathcal{U}}
\newcommand{\LTLStrongRelease}{\mathcal{M}}
\newcommand{\LTLGlobally}{\square}
\newcommand{\LTLFinally}{\lozenge}
\newcommand{\LTLSafety}{\phi_S}
\newcommand{\LTLLiveness}{\phi_L}
\newcommand{\LTLSafetyA}{\phi_{S1}}
\newcommand{\LTLLivenessA}{\phi_{L1}}

\newcommand{\Vertices}{V}
\newcommand{\Edges}{E}
\newcommand{\Root}{r}

\newcommand{\StatefulBehaviorTree}{\mathit{SBT}}
\newcommand{\SBTStates}{S_{SBT}}
\newcommand{\SBTInitialState}{s_{SBT}}
\newcommand{\SBTAlphabet}{\Sigma_{SBT}}
\newcommand{\SBTTransition}{\delta_{SBT}}
\newcommand{\StatusSet}{ST}
\newcommand{\Status}{st}
\newcommand{\VerticesPowerSeq}{VS}
\newcommand{\VerticesSeq}{vs}

\newcommand{\ttj}[1]{\textcolor{purple}{TTJ: #1}}

\definecolor{DSLCommentColor}{HTML}{778899}
\newcommand{\DSLComment}[2]{\hspace{#1}\textcolor{DSLCommentColor}{\text{\##2}}}
\newcommand{\SBTFSM}{BTSM}

\newcommand{\SimpleRobotX}{\BlVar{x}}
\newcommand{\SimpleRobotY}{\BlVar{y}}
%\newcommand{\SimpleRobotMission}{\BlVar{mis}}
\newcommand{\SimpleRobotGoalX}{\EnvVar{x_g}}
\newcommand{\SimpleRobotGoalY}{\EnvVar{y_g}}
\newcommand{\SimpleRobotMGoalX}{\BlVar{x_g}}
\newcommand{\SimpleRobotMGoalY}{\BlVar{y_g}}

% \newcommand{\SimpleRobotFlagX}{\EnvVar{x_f}}
% \newcommand{\SimpleRobotFlagY}{\EnvVar{y_f}}
% \newcommand{\SimpleRobotRobotX}{\EnvVar{x_r}}
% \newcommand{\SimpleRobotRobotY}{\EnvVar{y_r}}

\DeclareFloatingEnvironment[
  fileext   = loc,
  listname  = List of Grammars,
  name      = Grammar,
  placement = htbp,
]{Grammar}
